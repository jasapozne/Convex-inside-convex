\documentclass[a4paper, 12pt]{article}
\usepackage[utf8]{inputenc}
\usepackage[slovene]{babel}

\usepackage{amsthm}
\usepackage{amsmath, amssymb, amsfonts}
\usepackage{relsize}
\usepackage{graphicx}
\usepackage{etoolbox}
\usepackage{setspace}
\graphicspath{ {./Slike/} }
\usepackage[
top    = 3.cm,
bottom = 3.cm,
left   = 3.cm,
right  = 3.cm]{geometry}
\usepackage{hyperref}
\usepackage{mathtools}
\usepackage{authblk}
\usepackage{makecell}
\usepackage[nottoc]{tocbibind}

\newtheorem{definicija}{Definicija}
\newtheorem{posledica}{Posledica}

\begin{document}

\begin{titlepage}
    \begin{center}
        \textsc{\LARGE Univerza v Ljubljani}\\[0.5cm]
        {\Large Fakulteta za matematiko in fiziko}\\[3cm]
        {\large Finančni praktikum}\\[0.5cm]
        {\huge Največja konveksna množica znotraj konveksne množice}\\[10.0cm]
    \end{center}

    \begin{minipage}{0.4\textwidth}
		\begin{flushleft}
			\large
			\textit{Avtorja:}\\
			Jure Sternad \\
			Rok Rozman \\
            Jaša Pozne
		\end{flushleft}
	\end{minipage}
	~
	\begin{minipage}{0.4\textwidth}
		\begin{flushright}
			\large
			\textit{Mentorja:}\\
			prof. dr. Sergio \textsc{Cabello} \\
			doc. dr. Janoš \textsc{Vidali}
		\end{flushright}
	\end{minipage}
	
	\vfill\vfill\vfill 
	\begin{center}
	{\large{Ljubljana, \today}} 
    \end{center}
	\vfill 

\end{titlepage}

\tableofcontents

\newpage

\section{Navodilo}

Če imamo podana konveksna mnogokotnika $P$ in $Q$ v koordinatni ravnini,
 potem je problem odločanja ali se $P$ lahko preslika v $Q$ 
 linearen program (izvedljivosti). 
 Poleg tega je problem odločanja za koliko lahko $P$ največ povečamo,
  da je lahko v $Q$ , tudi linearen program.
   V primeru, da je $P$ disk, je to tudi linearen program.


\section{Opis problema}
\begin{definicija}
    Konveksen poligon $P$ je tak poligon, za katerega velja, da pri poljubni izbiri dveh točk $p$ in $q$
     iz poligona $P$, daljica $pq$, ki povezuje omenjeni točki v celoti leži v poligonu $P$.
\end{definicija}

\begin{definicija}
    Translacije so preslikave oblike $\tau(\vec{x}) = \vec{x} + \vec{a}$ za nek $a \in \mathbb{R}$.
\end{definicija}

\begin{definicija}
    Rotacije so preslikave oblike $\tau (\vec{x}) = R_{\phi}\vec{x} + \vec{a}$ za nek $\phi \in (0, 2\pi)$ in $\vec{a} \in \mathbb{R}$. 
    Takšna preslikava ustreza rotaciji za kot $\phi$ okoli točke v ravnini, ki je določena z enačbo $\tau (\vec{x}) = x$. 
\end{definicija}


\section{Nadaljnji potek dela}

Naša naloga je, da naredimo eksperimente, v katerih bomo
 poiskali največje možne kvadrate, diske, enakostranične trikotnike … ,
  ki jih lahko preslikamo tako, da so znotraj danega konveksnega
   mnogokotnika. Eksperimente bomo reševali s pomočjo linearnega 
   programiranja. Poleg tega bomo ločili primere, ko P lahko rotiramo;
    v tem primeru bomo ločili več različnih rotacij. 
Za reševanje problema bomo uporabljali programski jezik Sage.


\end{document}